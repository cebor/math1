\documentclass[11pt,a4paper,oneside]{article}

\usepackage{mathtools}
\usepackage{tikz}
\usepackage[all]{xy}
\usepackage{amsfonts}
\usepackage[hidelinks]{hyperref}
\usepackage[utf8]{inputenc}

\usepackage{fancyhdr}
\setlength{\headheight}{15.2pt}
\pagestyle{fancy}
 

\begin{document}
\title{Mathematik 1}
\author{Felix Itzenplitz\\
  Informatik\\
  \url{http://github.com/cebor/math1/}
\date{\today}
\maketitle
\thispagestyle{empty}
\newpage


% Kapitel 1
%%%%%%%%%%%%%%%%%%%%%%%%%%%%%%%%%%%%%%%%%%%%%%%%%%%%%%%%%%%%%%
\section{Elementare Logik}

\paragraph{Def.:}\mbox{}\\
Unter einer Aussage versteht man eine Zeichenreihe, der genau\\
eine der beiden Eigenschaften ``wahr'' ($w$) oder ``falsch'' ($f$) zukommt.


\paragraph{Bsp.:}

\begin{itemize}
  \item ``Die Erde ist ein Planet'' ($w$)
  \item 2 $\times$ 3 = 5 ($f$)
  \item ``wie geht es Dir?'' \emph{(keine Aussage)}
\end{itemize}


\paragraph{Def.:}\emph{Verknüpfung von Aussagen}\\
Seien P und Q Aussagen.

\begin{enumerate}
  \item \textsc{Negation:} $\neg$P \emph{``nicht P''}\\[5pt]
    \begin{tabular}{ c | c }
      P & $\neg$P \\
      \hline
      $w$ & $f$ \\
      $f$ & $w$ \\
    \end{tabular}

  \item \textsc{Konjunktion:} P $\land$ Q \emph{``P und Q''}\\[5pt]
    \begin{tabular}{c | c | c}
      P & Q & P$\land$Q \\
      \hline
      $w$ & $w$ & $w$ \\
      $w$ & $f$ & $f$ \\
      $f$ & $w$ & $f$ \\
      $f$ & $f$ & $f$ \\
    \end{tabular}

  \item \textsc{Disjunktion:} P $\lor$ Q \emph{``P oder Q''}\\[5pt]
    \begin{tabular}{c | c | c}
      P & Q & P$\lor$Q \\
      \hline
      $w$ & $w$ & $w$ \\
      $w$ & $f$ & $w$ \\
      $f$ & $w$ & $w$ \\
      $f$ & $f$ & $f$ \\
    \end{tabular}

  \item \textsc{Implikation:} P $\Rightarrow$ Q \emph{``Aus P folgt Q''}\\[5pt]
    \begin{tabular}{c | c | c}
      P & Q & P$\Rightarrow$Q \\
      \hline
      $w$ & $w$ & $w$ \\
      $w$ & $f$ & $f$ \\
      $f$ & $w$ & $w$ \\
      $f$ & $f$ & $w$ \\
    \end{tabular}

  \newpage

  \item \textsc{Äquivalenz:} P $\Leftrightarrow$ Q \emph{``P genau dann, wenn Q''}\\[5pt]
    \begin{tabular}{c | c | c}
      P & Q & P$\Leftrightarrow$Q \\
      \hline
      $w$ & $w$ & $w$ \\
      $w$ & $f$ & $f$ \\
      $f$ & $w$ & $f$ \\
      $f$ & $f$ & $w$ \\
    \end{tabular}
\end{enumerate}

\paragraph{Bsp.:}\mbox{}\\
  P: \emph{``FCB wird deutscher Meister''}\\
  Q: \emph{``FCB wird Pokal-Sieger''}\\
  R: \emph{``FCB steigt nicht ab''}

\paragraph{Dann ist:}\mbox{}\\
  \begin{tabular}{ll}
    $\neg$ R: & \emph{``FCB steigt ab''}\\
    P $\land$ Q: & \emph{``FCB wird dt. Meister und Pokal-Sieger''}\\
    P $\lor$ Q: & \emph{``FCB gewinnt min. einen Titel''}\\
    P $\Rightarrow$ R: & \emph{``Wenn FCB dt. Meister wird, dann steigt FCB nicht ab''}
  \end{tabular}

\paragraph{Bsp.:}
  \[\neg(P \land Q) \Leftrightarrow \neg P \lor \neg Q\]

\paragraph{Beweis:}
  \begin{center}
    \begin{tabular}{c | c | c | c | c | c | c | c}
      P & Q & P$\land$Q & $\neg$(P$\land$Q) & $\neg$P & $\neg$Q & $\neg$P$\lor\neg$Q & $\neg$(P$\land$Q)$\Leftrightarrow\neg$P$\lor\neg$Q \\
      \hline
      $w$ & $w$ & $w$ & $f$ & $f$ & $f$ & $f$ & $w$ \\
      $w$ & $f$ & $f$ & $w$ & $f$ & $w$ & $w$ & $w$ \\
      $f$ & $w$ & $f$ & $w$ & $w$ & $f$ & $w$ & $w$ \\
      $f$ & $f$ & $f$ & $w$ & $w$ & $w$ & $w$ & $w$ \\
    \end{tabular}
    \\[11pt]
    Die Aussage ist für alle P, Q wahr:\\
    $\Rightarrow$ allgemein gültige Aussage (Tautologie)
  \end{center}


\paragraph{Bsp.:}
  \[\neg(P \Rightarrow Q) \Leftrightarrow (P \land Q) \lor \neg P\]

\paragraph{Beweis:}
  \begin{center}
    \begin{tabular}{c | c | c | c | c | c | c}
      P & Q & P$\Rightarrow$Q & P$\land$Q & $\neg$P & (P$\land$Q)$\land\neg$P & $\neg$(P$\Rightarrow$Q)$\Leftrightarrow$(P$\land$Q)$\lor\neg$P \\
      \hline
      $w$ & $w$ & $w$ & $w$ & $f$ & $w$ & $w$ \\
      $w$ & $f$ & $f$ & $f$ & $f$ & $f$ & $w$ \\
      $f$ & $w$ & $w$ & $f$ & $w$ & $w$ & $w$ \\
      $f$ & $f$ & $w$ & $f$ & $w$ & $w$ & $w$ \\
    \end{tabular}
  \end{center}

\paragraph{}\mbox{}\\
  Häufig hat man Aussagen, in denen eine ``Objekt-Variable'' vorkommt, z.B.:\\
  P(x): \emph{``x hat rote Haare''}\\
  Q(x): \emph{``x ist eine ganze Zahl''}\\
  Fehlende Angaben: woher ist \textbf{x} und für welche \textbf{x} gilt es.\\[8pt]
  Formulierung über \emph{Quantoren}:

\paragraph{Def.:}
  \begin{enumerate}
    \item $\forall x \in M$: P(x) ``Für alle x aus M gilt P(x).''
    \item $\exists x \in M$: P(x) ``Es gibt min. ein x aus M, sodass P(x) gilt.''
  \end{enumerate}

\paragraph{Bsp.:}\mbox{}\\
  $M = \text{Studenten FH-Rosenheim}$\\
  $P(x)$ - ``x hat rote Haare''\\
  $\exists x \in M: P(x)$ ``es gibt einen Student mit roten Haaren an der FH''

\paragraph{Bsp.:}\mbox{}\\
  $M = \text{alle Menschen}$\\
  $P(x)$ - ``x ist glücklich''\\
  $\forall x \in M: P(x)$ ``alle Menschen sind glücklich''\\[5pt]
  Negation: ``Es gibt einen Menschen der nicht glücklich ist''\\
  $\Rightarrow$ $\exists x \in M: \neg P(x)$

\paragraph{Satz:}
  \begin{enumerate}
    \item $\neg(\forall x \in M: P(x)) \Leftrightarrow \exists x \in M: \neg P(x)$
    \item $\neg(\exists x \in M: P(x)) \Leftrightarrow \forall x \in M: \neg P(x)$
  \end{enumerate}
  Hat man Aussagen, die von mehreren ``Objekt-Variablen'' abhängen. z.B.
  \[P(x,y):\:(x-y)^2\geq 0\]
  So muss man die Variablen x,y spezifizieren ($\Rightarrow$ Quantoren)
  und dabei spielt es eine Rolle in welcher Reihenfolge sie auftreten.\\[5pt]
  Dabei gilt es folgende Regel zu beachten:

  \begin{enumerate}
    \item Zwei benachbarte gleiche Quantoren kann man vertauschen:\\[3pt]
      \begin{tabular}{clc}
        & $\forall x \in \mathbb{R}\;\;\;\;\forall y \in \mathbb{R}:$ & $(x-y)^2\geq 0$ \\
        $\Leftrightarrow$ & $\forall y \in \mathbb{R}\;\;\;\;\forall x \in \mathbb{R}:$ & $(x-y)^2\geq 0$ \\
        $\Leftrightarrow$ & $\forall x,y \in \mathbb{R}:$ & $(x-y)^2\geq 0$ \\
      \end{tabular}
    \\[3pt]Für $\exists$ analog.
    \item Verschiedene Quantoren dürfen nicht vertauscht werden.\\[3pt]
      $\forall x \in \mathbb{R}\;\;\;\;\exists n \in \mathbb{N}:\;\;n>x\;\;\;\;\;\;\;\;(w)$\\
      $\exists n \in \mathbb{N}\;\;\;\;\forall x \in \mathbb{R}:\;\;n>x\;\;\;\;\;\;\;\;(f)$

  \end{enumerate}

\paragraph{Bsp:}\mbox{}\\
  \hangindent=0.5cm
  $M = \{Klitschko, Ali, Tyson\}$ \\
  $P(x,y):$ ``x gewinnt gegen y''


  \subparagraph{Szenario 1:}
    \begin{displaymath}
      \xymatrix{ & Klitschko \ar@/_/[dl] \ar@/^/[dr] & \\
                 Tyson & & Ali \ar@/^/[ll] }
    \end{displaymath}

  \subparagraph{Szenario 2:}
    \begin{displaymath}
      \xymatrix{ & Klitschko \ar@/^/[dr] & \\
                 Tyson \ar@/^/[ur] & & Ali \ar@/^/[ll] }
    \end{displaymath}


  \subparagraph{}
    \begin{displaymath}
      \forall x \in M \; \exists x \neq y \in M: \text{ ``x gewinnt gegen y''}
    \end{displaymath}
    \begin{center}
      $\Rightarrow$ Szenario 1: ($f$), Szenario 2: ($w$)\\[5pt]
    \end{center}

    \begin{displaymath}
      \exists x \in M \; \forall x \neq y \in M: \text{ ``x gewinnt gegen alle''}
    \end{displaymath}
    \begin{center}
      $\Rightarrow$ Szenario 1: ($w$), Szenario 2: ($f$)\\[5pt]
    \end{center}

\newpage

% Kapitel 2
%%%%%%%%%%%%%%%%%%%%%%%%%%%%%%%%%%%%%%%%%%%%%%%%%%%%%%%%%%%%%%
\section{Elementare Mengenlehre}

\paragraph{Def.:}(Cantor)\\
  Eine Menge ist eine Zusammenfassung von bestimmten und wohlunterschiedenen Objekten
  unserer Aussuchung oder unseres Denkens zu einem Ganzen. (Objekte heißen Elemente)

\paragraph{Bsp.:}
  \begin{enumerate}
    \item $M = \{-1, 0, 1, 2\} = \{0, 1, 2, -1\} = \{-1, -1, 0, 1, 2\}$ 
    \item $M_1 = \{2x + 1 | x \in M \} = \{-1, 1, 3, 5\}$
    \item leere Menge $\{\}$
    \item Zahlen: $\mathbb{N}, \mathbb{Z}, \mathbb{Q}, \mathbb{R}, \mathbb{C}$
  \end{enumerate}

\paragraph{Def.:}\mbox{}\\
  Eine Menge $A$ heißt \emph{Teilmenge} von $B$ in Symbolen:

  \[A \subset B:\:\Leftrightarrow \forall x \in A: \: x \in B \]

\paragraph{Bsp.:}
  \begin{enumerate}
    \item $M \not\subset M_1$, da $0 \in M$, aber $0 \notin M_1$
    \item $\emptyset \subset M$, da $\emptyset$ enthält kein Element.
    \item $\{1,2\} \subset \{1,2,3,\mathbb{N}\}$
  \end{enumerate}

\paragraph{Def.:} Seien A, B Mengen
  \begin{enumerate}
    \item \emph{Durchnitt}: $A \cap B:=\{x|x \in A \land x \in B\}$
    \item \emph{Vereinigung}: $A \cup B:=\{x|x \in A \lor x \in B\}$
    \item \emph{Kompliment}: $A \setminus B:=\{x|x \in A \land x \in B\}$
  \end{enumerate}

\paragraph{Bsp.:}
  \[A=\{1,2,3\}, B=\{2,3,4,5\}, C=\{2,5,6\}\]

  \begin{enumerate}
    \item $A \cup B = \{1,2,3,4,5\}$
    \item $A \cap B = \{2,3\}$
    \item $A \setminus B = \{1\}$
    \item $B \setminus A = \{4,5\}$
    \item $C \setminus (A \cap B) = \{5,6\}$
    \item $(C \setminus A) \cap B = \{5\}$
  \end{enumerate}

\paragraph{Def.:}\mbox{}\\
  Seien $A_1, \ldots, A_n$ Mengen $\neq \emptyset$, dann heißt
  \[A_1 \times A_2 \times \ldots \times A_n := \{(a_1, \ldots, a_n) | \forall \: 1 \leq i \leq n: \: a_i \in A_i\}\]
  \emph{kartesisches Produkt} der $A_i$.\\[10pt]
  Eine Element $(a_1, \ldots, a_n)$ heißt \emph{n-Tupel}. Ist $A_1 = A_2 = \ldots = A_n$, so schreibt man
  $A_1^n = A_1 \times \ldots \times A_n$

\paragraph{Bsp.:}
  \begin{enumerate}
    \item $M_1 = \{1, 2\}, \: M_2=\{3, 4, 5\}$
      \[M_1 \times M_2 = \{(1,3), (1,4), (1,5), (2,3), (2,4), (2,5)\}\]
    \item $M = \{1,2,3\}$.
      \[M^2 = M \times M = \{(1,1), (1,2), (1,3), (2,3), (2,1), (2,2), (3,1), (3,2), (3,3)\}\]
  \end{enumerate}

\paragraph{Def.:}\mbox{}\\
  Eine \emph{Abbildung} oder Funktion $f$ von einer Menge $D$ in eine Menge $M$ ist
  eine Vorschrift, die jedem $x \in D$ genau ein Element $f(x) \in M$ zuordnet.
  \[f: \: D \to M\]
  \[x \mapsto f(x)\]
  d. h. $\forall x \in D \: \exists_1 y \in M: \: f(x) = y$

\paragraph{Bsp.:}\mbox{}\\
  \begin{enumerate}
    \item $f: \: \mathbb{N} \to \mathbb{N}$\\
    $n \mapsto f(n):= n^2$\\[10pt]
    \[f(1) = 1^2 = 1\]
    \[f(2) = 2^2 = 4\]
    \[\vdots\]\\
    \item ASCII-Zeichensatz: $f:\{Zeichen\}\to\{0,\ldots,127\}$\\
    \[A \mapsto 65\]
    \[B \mapsto 66\]
    \[\$ \mapsto 36\]\\
    \item $f: \mathbb{R}^+_0 = [0,\infty) \to \mathbb{R}$
    \[x \mapsto f(x):= \sqrt{x}\]
  \end{enumerate}


\end{document}
